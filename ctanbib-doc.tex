\documentclass{ltxdoc}
\newcommand\package[1]{\texttt{#1}}
\usepackage{longtable}
\usepackage{tabularx}
\newenvironment{changelog}{\longtable{@{} l p{30em}}}{\endlongtable}
\newcommand\change[2]{#1 & #2\\}
\title{The \texttt{ctanbib} package}
\author{Michal Hoftich\\\url{michal.h21@gmail.com}}
\date{\gitdate, version \gitversion}
\usepackage{hyperref}
\usepackage{fontspec}
\setmainfont{TeX Gyre Schola}
\setmonofont[Scale=MatchLowercase]{Inconsolatazi4}
\usepackage{microtype}
\begin{document}
\maketitle

\section{Introduction}
This package provides script named \package{ctanbib} which can be used for retrieving of the bibliographic information for packages hosted on CTAN.


\section{Usage}


It can be used in the following way:


\noindent\package{ctanbib [options]  $\langle$packagename$\rangle$ > filename.bib}

The \texttt{curl} utility needs to be installed on the user's system in order to support the package info download.


\section{Available options}

\begin{description}
  \item[-c,-\/-ctan] use the \texttt{@ctan} type instead of \texttt{@manual}.
    The difference between the two is that the url field contains packages CTAN
    path, instead of a link to the package.
  \item[-h,-\/-help] print the help message.
  \item[-v,-\/-version] print the version info.
\end{description}


\section{Examples}

\begin{verbatim}
ctanbib latex  
\end{verbatim}


\noindent This invocation will print data in the Bib\LaTeX\ format to the standard output:

\begin{verbatim}
@manual{latex,
title = {The Latex package},
subtitle = {A TeX macro package that defines LaTeX},
author = {Lamport, Leslie and The LaTeX Team},
url = {http://www.latex-project.org/},
urldate = {2018-08-23}, 
date = {2018-04-01},
version = {PL 5}
}
\end{verbatim}

The \texttt{--ctan} option:

\begin{verbatim}
ctanbib -c hyperref
\end{verbatim}

\noindent This produces a bib record with the \texttt{@ctan} type:

\begin{verbatim}
@ctan{hyperref,
title = {The Hyperref package},
subtitle = {Extensive support for hypertext in LaTeX},
author = {Rahtz, Sebastian and Oberdiek, Heiko},
url = {/macros/latex/contrib/hyperref},
urldate = {2018-08-27}, 
date = {},
version = {6.86b}
}
\end{verbatim}


\section{License}

Permission is granted to copy, distribute and/or modify this software
under the terms of the \LaTeX\ Project Public License, version 1.3.

\section{Acknowledgements}

This package was created thanks to TeX.sx user
Daniel\footnote{\url{https://tex.stackexchange.com/users/3751/daniel}}. Lukas
C.~Bossert had useful comments which led to update of the original script and
it's eventual release on CTAN.

\newpage
\section{Changelog}

\begin{changelog}
  \change{2019-09-16}{Use the CTAN API 2.0 to fetch the package info}
  \change{2019-05-04}{Version 0.1c released}
  \change{2019-05-04}{Use the \texttt{curl} command to download the package info}
  \change{2019-05-04}{Added information about the need to have the \texttt{curl} command installed}
  \change{2019-03-11}{Version 0.1b released}
  \change{2018-09-10}{Added support for institutional names}
  \change{2018-09-10}{Escape TeX commands in the title and subtitle}
  \change{2018-08-29}{Added man page}
  \change{2018-08-27}{Added --ctan option}
  \change{2018-08-23}{Version 0.1 released.}
  \change{2018-08-23}{Created documentation.}
  \change{2018-08-21}{Rewrote the script to use the DOM api provided by LuaXML. This provides greater flexibility in the XML processing.}
  \change{2018-08-20}{Fixed the script for the current XML format provided by CTAN.}
  \change{2014-09-12}{Initial version of the script have been released in answer on TeX.sx\footnote{\url{https://tex.stackexchange.com/a/200856/2891}}.}
\end{changelog}


\end{document}
